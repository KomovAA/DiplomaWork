\intro
%Рассеяние электромагнитных волн было предметом исследований с начала века. Традиционно рассеивающие объекты, которые являются большими по сравнению с длиной волны, могут быть точно смоделированы с помощью приближенного высокочастотного метода, такого как геометрическая оптика или физическая оптика. 
%Действительно, для определенного класса геометрий геометрическая и физическая оптика вполне подходят, в то время как геометрическая теория дифракции или физическая теория дифракции могут использоваться для уточнения такой модели, когда присутствуют ребра, разрывы кривизны или соединения материалов. Для объектов, которые малы по сравнению с длиной волны, полезны низкочастотные методы, такие как приближение Борна. К сожалению, низко- и высокочастотные методы предъявляют довольно жесткие требования к размеру, ориентации и составу рассеивающего объекта.  Для функций размера волны мы обычно прибегаем к методологии численного решения. \\
%Альтернативный подход, который в последнее время оказался достаточно мощным для приложений рассеяния - это метод конечных элементов. Поскольку этот метод является подходом с уравнением в частных производных, он получает преимущество от свойства локальности, присущего всем формулировкам в частных производных. Формула метода конечных элементов моделирует неоднородные диэлектрики и металлические поверхности довольно хорошо, в отличии от других способов. \\
%Метод является многообещающим подходом при моделировании больших композитных структур в контексте единой надежной формулировки. Однако, как и в случае с каждым методом уравнения в частных производных, схема усечения сетки играет важную роль в оценке точности и эффективности реализации. Точные схемы усечения сетки имеют недостаток, заключающийся в разрушении разреженности матрицы конечных элементов, но в большинстве случаев они также уменьшают вычислительную область. Приближенные схемы усечения основаны на использовании поглощающих граничных условий или поглощающих материалов, целью которых является подавление отражений волн обратно в вычислительную область.  В этой работе будет рассмотренно рассеяние плоской электромагнитной волны на рассеивателях, вычисленное с помощью метода конечных элементов и поглощающих граничных условий.

Как известно, все, что нас окружает, мы видим благодаря свету, отраженному или рассеянному от объектов. Свет --- ничто иное, как электромагнитное излучение, такое же как рентгеновские лучи или радиоволны. Поняв со временем природу радиоизлучения, пришла возможность управлять им варьируя характеристики антенн. Подобно этому, изменяя параметры среды в различных диапазонах частот появляется возможность контролировать поведение пучков света. Реализовать подобное можно с помощью материалов, контролирующих ход лучей --- метаматериалов, варьирую геометрические характеристики которых, можно управлять показателем преломления среды. 
Такие необычные среды в 1967 году предсказал советский физик Виктор Георгиевич Веселаго~\cite{b11}. Он утверждал, что материалы могут иметь такие особые оптические свойства, как отрицательный показатель преломления и обратный эффект Доплера - так называемые, левые среды. В доказательство его теории подобные среды появились только в 2000 году. Если обычно, свойства материала характеризуется веществом, из которого он состоит, то в метаматериалах их определяет геометрия. Отрицательного коэффициента преломления можно достигнуть, если диэлектрическая проницаемость вместе с магнитной проницаемостью будут отрицательными. Однако это будет верно лишь в небольшом диапазоне длин волн, который в свою очередь будет зависеть от габаритов элементарного элемента, образующего метаматериал и свойств, заполняющей его среды. Отсюда вытекает, что, уменьшая размер антенны, мы уменьшаем длину волны, для которой среда будет иметь отрицательный показатель преломления. Отличительно особенностью современных метаматериалов со свойствами левых сред является относительно большие потери в диапазоне частот, в котором эти среды имеют отрицательный коэффициент преломления. Для большинства приложений в оптическом диапазоне частот такая особенность является отрицательной характеристикой среды, но, она может иметь положительный эффект в радиодиапазоне для нужд уменьшения отражательной способности поверхностей объектов.

В последние годы активно развиваются так называемые фрактальные антенны. Это антенны, главным образом отличающиеся своей геометрией, при конструировании которых используются фракталы или квазифрактальные структуры, проявляющиеся в рекурсивном повторении изначальных шаблонов, для максимизации эффективной длины волны в ограниченном пространстве. В 1986 году ученые Университета штата Пенсильвания Я. Ким и Д. Джаггард впервые опубликовали свои работы, в которых упоминаются фрактальные структуры \cite{b1}. Первые теоретические исследования приписывают Каталонии К. Пуенте, который писал о возможностях применения фрактальных структур для конструирования многополосных антенн \cite{b8},\cite{b9}. Принято считать, что практическое применение фрактальных антенн заложил американский инженер Натан Коэн~\cite{b10}. Фрактальные антенны, ввиду экономии и более эффективного использования заданного ограниченного пространства, имеют явное преимущество перед антеннами, созданными на основе евклидовой геометрии. Описанные особенности фрактальных антенн дают надежду на улучшение требуемых характеристик метаматериалов, в случае использования фрактальных объектов в качестве элементов композитной среды.

%Чтобы добиться того, что облучаемый антенной объект станет незаметным, метаматериал должен иметь внедрение структуры меньше длины волны. Например, для фиолетового света с длиной волны 400 нм, внедренные структуры длиной около 40 нм. Для изменения хода светового луча, мы должны получить отрицательные значения диэлектрической и магнитной проницаемости среды, которые в свою очередь, получаются, когда силы, создаваемые электрическими и магнитными полями, воздействуют в противоположном направлении движению электронов в материале. Чтобы добиться подобной отрицательной реакции, необходимо подбирать частоту колебания при использовании резонансной характеристики среды.

Цель настоящей работы --- исследовать такие интегральные характеристики элементарных рассеивателей при рассеянии на них плоской волны как сечение рассеяния и сечение поглощения, в зависимости от сложности геометрии объекта, материалов, составляющих элементы, и частоты падающего излучения.
В начале рассмотрим рассеяние плоской электромагнитной волны на рассеивателях в различных диапазонах частот для шариков, состоящих из серебра, меди и золота. Для этого представим шарики в виде сфер для трех случаев: для одиночной сферы, для 5 сфер и для 17 сфер. Решим задачу дифракции одиночной сферы используя метод конечных элементов, и применим тот же подход для остальных случаев. А также найдем рассеяние на цилиндре численным и точным методом решения, чтобы убедиться в правильности получаемых данных. Протестируем полученные численные схемы и проанализируем полученные результаты.
