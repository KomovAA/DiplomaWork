\intro
Рассеяние электромагнитных волн было предметом исследований с начала века. Традиционно рассеивающие объекты, которые являются большими по сравнению с длиной волны, могут быть точно смоделированы с помощью приближенного высокочастотного метода, такого как геометрическая оптика или физическая оптика. 
Действительно, для определенного класса геометрий геометрическая и физическая оптика вполне подходят, в то время как геометрическая теория дифракции или физическая теория дифракции могут использоваться для уточнения такой модели, когда присутствуют ребра, разрывы кривизны или соединения материалов. Для объектов, которые малы по сравнению с длиной волны, полезны низкочастотные методы, такие как приближение Борна. К сожалению, низко- и высокочастотные методы предъявляют довольно жесткие требования к размеру, ориентации и составу рассеивающего объекта.  Для функций размера волны мы обычно прибегаем к методологии численного решения. \\
Альтернативный подход, который в последнее время оказался достаточно мощным для приложений рассеяния, - это метод конечных элементов. Поскольку этот метод является подходом с уравнением в частных производных, он получает преимущество от свойства локальности, присущего всем формулировкам в частных производных. Формула метода конечных элементов моделирует неоднородные диэлектрики и металлические поверхности довольно хорошо, в отличии от других способов. \\
Метод является многообещающим подходом при моделировании больших композитных структур в контексте единой надежной формулировки. Однако, как и в случае с каждым методом уравнения в частных производных, схема усечения сетки играет важную роль в оценке точности и эффективности реализации. Точные схемы усечения сетки имеют недостаток, заключающийся в разрушении разреженности матрицы конечных элементов, но в большинстве случаев они также уменьшают вычислительную область. Приблизительные схемы усечения основаны на использовании поглощающих граничных условий или поглощающих материалов, целью которых является подавление отражений волн обратно в вычислительную область. Они сохраняют разреженность матрицы конечных элементов, но требуют, чтобы мы увеличили вычислительную область, чтобы обеспечить приемлемые уровни точности. В этой работе мы рассмотрим метод конечных элементов и поглощающие граничные условия для вычислений рассеяния.