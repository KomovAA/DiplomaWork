\intro
%Рассеяние электромагнитных волн было предметом исследований с начала века. Традиционно рассеивающие объекты, которые являются большими по сравнению с длиной волны, могут быть точно смоделированы с помощью приближенного высокочастотного метода, такого как геометрическая оптика или физическая оптика. 
%Действительно, для определенного класса геометрий геометрическая и физическая оптика вполне подходят, в то время как геометрическая теория дифракции или физическая теория дифракции могут использоваться для уточнения такой модели, когда присутствуют ребра, разрывы кривизны или соединения материалов. Для объектов, которые малы по сравнению с длиной волны, полезны низкочастотные методы, такие как приближение Борна. К сожалению, низко- и высокочастотные методы предъявляют довольно жесткие требования к размеру, ориентации и составу рассеивающего объекта.  Для функций размера волны мы обычно прибегаем к методологии численного решения. \\
%Альтернативный подход, который в последнее время оказался достаточно мощным для приложений рассеяния - это метод конечных элементов. Поскольку этот метод является подходом с уравнением в частных производных, он получает преимущество от свойства локальности, присущего всем формулировкам в частных производных. Формула метода конечных элементов моделирует неоднородные диэлектрики и металлические поверхности довольно хорошо, в отличии от других способов. \\
%Метод является многообещающим подходом при моделировании больших композитных структур в контексте единой надежной формулировки. Однако, как и в случае с каждым методом уравнения в частных производных, схема усечения сетки играет важную роль в оценке точности и эффективности реализации. Точные схемы усечения сетки имеют недостаток, заключающийся в разрушении разреженности матрицы конечных элементов, но в большинстве случаев они также уменьшают вычислительную область. Приближенные схемы усечения основаны на использовании поглощающих граничных условий или поглощающих материалов, целью которых является подавление отражений волн обратно в вычислительную область.  В этой работе будет рассмотренно рассеяние плоской электромагнитной волны на рассеивателях, вычисленное с помощью метода конечных элементов и поглощающих граничных условий.

Видимый пучок света — это одна из форм электромагнитного излучения, как радиоволны и рентгеновские лучи, гамма-кванты. Подобно тому, как мы умеем управлять радиоизлучением с помощью антенн, мы также можем изменять поведение пучка света других диапазонов. Все, что видит наш глаз, — это свет, «испущенный», или, точнее, рассеянный или отраженный предметами. Качественное чистое стекло практически не рассеивает и не отражает свет — поэтому нам не составляет труда смотреть в окно, не обращая внимания на стеклянную преграду. Увидеть же предмет можно благодаря тому, что он искажает ход лучей, которое его окружает. Препятствовать этому могло бы устройство, которое «восстанавливает» световое поле и словно бы заставляет лучи света огибать предмет. Для этого нам необходимы материалы, позволяющие идеально контролировать распространение света - метаматериалы, которые способны управлять показателем преломления.
Необычные свойства метаматериалов в 1967 году предсказал советский физик Виктор Георгиевич Веселаго (сотрудник Института общей физики имени Александра Михайловича Прохорова РАН, Москва). Он показал, что материалы могут обладать необычными оптическими свойствами, как отрицательный показатель преломления и обратный эффект Доплера.
Однако лишь в 2000 году физикам впервые удалось доказать, что среды с отрицательным коэффициентом преломления действительно существуют. В отличие от классических материалов, свойства которых определяются в основном веществом, из которого они состоят, свойства метаматериалов определяются их геометрией. Чтобы добиться отрицательного преломления, необходимо, чтобы отрицательными были сразу два свойства материала — диэлектрическая проницаемость и магнитная восприимчивость.
Однако такой материал работает лишь в очень узком диапазоне длин волн, который напрямую определяется размерами и формой антенн. Чем меньше размеры антенны, тем меньше и длина волны, для которой среда имеет отрицательный коэффициент преломления. 
Фрактальные антенны – относительно новый класс электрически малых антенн, принципиально отличающийся своей геометрией. Главное отличие фрактальных геометрических форм – их дробная размерность, что внешне проявляется в рекурсивном повторении в возрастающем либо уменьшаемом масштабах исходных детерминированных или случайных шаблонов. Фрактальные технологии получили распространение при формировании средств фильтрации сигналов, синтезе трехмерных компьютерных моделей природных ландшафтов, сжатии изображений.
Первые публикации по электродинамике фрактальных структур относятся к 80-м годам прошлого века, в которых упоминается работа ученых Университета штата Пенсильвания Я.Кима
и Д.Джаггарда (Y.Kim and D.L.Jaggard). Первенство в теоретических исследованиях возможности применения фрактальных форм для формирования многополосных по частоте антенн приписывают ученому Технологического университета Каталонии К.Пуенте (C.Puente). Начало же практическому применению фрактальных антенн в 1995 году положил, как принято считать, американский инженер Натан Коэн
(N.Cohen). Возможности фракталов, такие как экономия пространства и эффективное использование всего ограниченного объема, обеспечивают явное преимущество интегрированных фрактальных антенн по сравнению с теми, которые созданы на основании Евклидовой геометрии. 
Чтобы сделать облучаемый антенной объект невидимым, метаматериал должен иметь внедрение структуры меньше длины волны. Например, для зеленого света с длиной волны 500 нм, внедренные структуры длиной около 50 нм. Чтобы произвольно искривлять путь светового луча, нам необходимо модифицировать отдельные атомы. Отрицательные значения диэлектрической и магнитной проницаемостей среды получаются в том случае, когда электроны в материале движутся в направлении, противоположном по отношению к силам, создаваемым электрическим и магнитным полями. Ключ к такого рода отрицательной реакции — использование резонансной характеристики среды, выбор колебаний со специфической частотой. 
Цель работы - исследовать такие интегральные характеристики рассеивателей при рассеянии на них плоской волны как сечение рассеяния и сечение полглощения, в зависимости от сложности геометрии объекта, материлов, состовляющих элементы, и частоты падающего излучения. 
В начале рассмотрим рассеяние плоской электромагнитной волны на рассеивателях в различных диапазонах частот для шариков, состоящих из серебра, меди и золота. Для этого представим шарики ввиде сфер для трех случаев, для одиночной сферы, для 5 сфер и для 17 сфер. Решим задачу дифракции одиночной сферы используя метод конечных элементов, и применим тот же подход для остальных случаев. А так же найдем рассеянние на цилиндре численным и точным методом решения, чтобы убедиться в правильности получаемых данных. Протестируем полученные численные схемы и проанализируем полученные результаты.