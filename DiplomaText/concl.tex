\conclusion
В настоящей работе рассмотрены такие интегральные характеристики рассеивателей при рассеянии на них плоской волны как сечение рассеяния и сечение поглощения, при разной геометрии объекта, материалов, составляющих элементы, и частоты падающего излучения. Получены выражения для рассеяния плоской электромагнитной волны на одиночном цилиндре численным и точным методом. Рассмотрена задача дифракции одиночной сферы используя метод конечных элементов. Были протестированы полученные численные схемы с помощью программы FreeFem. Установлены зависимости частот от длин волн при рассеянии на разных типах и колличествах сфер. А именно, получены графики для сечения поглощения и сечения рассеяния на одиночном медном,серебряном,золотом шаре. Их первой итерации состоящей из 5 шаров, а так же для второй инетарции состоящей из 17 шаров. Показано, что увеличение сложности рассеивателя приводит к увеличению сечения рассеяния для всех сред заполнения и всех длин волн, за исключением небольшого интервала в оптической области частот для серебряных рассеивателей в котором семнадцать шаров имеют меньшее сечение рассеяния, чем рассеиватель из пяти шаров.
В тоже время, сечения поглощения для всех материалов в инфракрасной области частот возрастает с увеличением числа элементов рассеивателя, а в оптической области сечение поглощения рассеивателя из семнадцати шаров уменьшается по сравнению с сечением поглощения рассеивателя из пяти шаров.
При этом установлено, что сечение поглощения и сечение рассеяния имеют наименьшие значения практически на всех длинах волн для рассеивателей, составленных из серебра. 
Показано, что для серебра и золота, локальный минимум зависимости сечения поглощения от длины волны смещается в сторону больших длин волн с увеличением сложности рассеивателя.

