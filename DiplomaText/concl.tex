\conclusion
В настоящей работе рассмотрены такие интегральные характеристики рассеивателей при рассеянии на них плоской волны как сечение рассеяния и сечение поглощения, при разной геометрии объекта, материалов, составляющих элементы, и частоты падающего излучения. Получены выражения для рассеяния плоской электромагнитной волны на одиночном цилиндре численным и точным методом. Рассмотрена задача дифракции одиночной сферы используя метод конечных элементов. Были протестированы полученные численные схемы с помощью программы FreeFem. Установлены зависимости частот от длин волн при рассеянии на разных типах и колличествах сфер. А именно, получены графики для поглощения и рассеяния электромагнитной волны на одиночной медной,серебряной,золотой сфере. Их первой итерации состоящей из 5 сфер, а так же для второй инетарции состоящей из 17 сфер. Было показано, что некоторые графики имеют схожую зависимость с гиперболическим косекансом. Установлено, что с ростом частоты происходит уменьшение диапазона длин волн как для сечения рассеяния, так и для сечения поглощения. Увеличении элементов рассеивателя приводит к расширению диапозона длин волн, однако с каждой итерацией это изменение становится менее значительно. Было показано, как именно выбор материала может повлиять на максимальную частоту и диапозон длин волн. 
