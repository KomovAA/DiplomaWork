\conclusion
В настоящей работе исследованы характеристики фрактальных объектов в широком диапазоне частот при рассеянии на них плоской волны. Были рассмотрены такие интегральные характеристики рассеивателей как сечение рассеяния и сечение поглощения, при разной геометрии объекта, материалов, составляющих элементы, и частоты падающего излучения. Получены выражения для рассеяния плоской электромагнитной волны на одиночном цилиндре численным и точным методом. Были протестированы полученные численные схемы с помощью программы FreeFem. Получены результаты для таких конфигураций рассеивателей как одиночный шар, набор из 5 шаров, а так же рассеиватель, состоящий из 17 шаров. Показано, что увеличение сложности рассеивателя приводит к увеличению сечения рассеяния для всех сред заполнения и всех длин волн, за исключением небольшого интервала в оптической области частот для серебряных рассеивателей в котором семнадцать шаров имеют меньшее сечение рассеяния, чем рассеиватель из пяти шаров.
В тоже время, сечения поглощения для всех материалов в инфракрасной области частот возрастает с увеличением числа элементов рассеивателя, а в оптической области сечение поглощения рассеивателя из семнадцати шаров уменьшается по сравнению с сечением поглощения рассеивателя из пяти шаров.
При этом установлено, что сечение поглощения и сечение рассеяния имеют наименьшие значения практически на всех длинах волн для рассеивателей, составленных из серебра. 
Показано, что для серебра и золота, локальный минимум зависимости сечения поглощения от длины волны смещается в сторону больших длин волн с увеличением сложности рассеивателя.

