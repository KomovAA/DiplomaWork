\documentclass[%
specialist,  % тип документа
%natbib,      % использовать пакет natbib для "сжатия" цитирований
subf,        % использовать пакет subcaption для вложенной нумерации рисунков
href,        % использовать пакет hyperref для создания гиперссылок
colorlinks,  % цветные гиперссылки
%fixint,     % включить прямые знаки интегралов
]{disser}

\usepackage[
  a4paper, mag=1000,
  left=2.5cm, right=1cm, top=2cm, bottom=2cm, headsep=0.7cm, footskip=1cm
]{geometry}

\usepackage[intlimits]{amsmath}
\usepackage{amssymb,amsfonts}

\usepackage[T2A]{fontenc}
\usepackage[utf8]{inputenc}
\usepackage[english,russian]{babel}
\ifpdf\usepackage{epstopdf}\fi
\usepackage[autostyle]{csquotes}

% Шрифт Times в тексте как основной
%\usepackage{tempora}
% альтернативный пакет из дистрибутива TeX Live
%\usepackage{cyrtimes}

% Шрифт Times в формулах как основной
%\usepackage[varg,cmbraces,cmintegrals]{newtxmath}
% альтернативный пакет
%\usepackage[subscriptcorrection,nofontinfo]{mtpro2}

% Плавающие рисунки "в оборку".
\usepackage{wrapfig}

\usepackage[style=gost-numeric,
  backend=biber,
  language=auto,
  hyperref=auto,
  autolang=other,
  sorting=none
]{biblatex}

\addbibresource{thesis.bib}

%\renewcommand{\chaptername}{}
%\renewcommand{\bibname}{Литература}
%\renewcommand{\figurename}{Рисунок}

\graphicspath{{fig/}}
\renewcommand{\thechapterfont}{\normalsize\bfseries} % Номер главы полужирным
\renewcommand{\prethechapter}{} % Убираем слово "глава"
%\renewcommand{\postthechapter}{.~} % ставим точку и пробел после номер
%\renewcommand{\appendixfont}{\normalsize\bfseries}
%\renewcommand{\chapterfont}{\normalsize\bfseries}
%\renewcommand{\sectionfont}{\normalsize\bfseries}
%\renewcommand{\subsectionfont}{\normalsize\bfseries}
%\renewcommand{\subsubsectionfont}{\normalsize\bfseries}
%\renewcommand{\tocprethechapter}{} % в оглавлении убираем слово "Глава"
%\renewcommand{\theappendixalign}{\hfill} % выключка вправо для слова "приложение" в приложениии
%\renewcommand{\theappendix}{\arabic{chapter}} % заменяем нумерацию приложений на цифры
%\renewcommand{\pretheappendix}{\protect{ПРИЛОЖЕНИЕ}~} % Меняем регистр слова "Приложение"
%\renewcommand{\tocpretheappendix}{\protect{ПРИЛОЖЕНИЕ}~} % Меняем регистр слова "Приложение"
\renewcommand{\introname}{ВВЕДЕНИЕ}
%\renewcommand{\sectionindent}{1cm}
%\renewcommand{\subsectionindent}{1cm}
%\renewcommand{\aftersection}{6pt plus .1pt}
%\renewcommand{\aftersubsection}{3pt plus .1pt}
\renewcommand{\conclusionname}{ЗАКЛЮЧЕНИЕ}
%\renewcommand{\bibintoc}{Литература}

% Номера страниц снизу и по центру
%\pagestyle{footcenter}
%\chapterpagestyle{footcenter}

% Точка с запятой в качестве разделителя между номерами цитирований
%\setcitestyle{semicolon}

% Использовать полужирное начертание для векторов
\let\vec=\mathbf

% Включать подсекции в оглавление
\setcounter{tocdepth}{2}

\graphicspath{{fig/}}

\begin{document}

% Переопределение стандартных заголовков
\def\contentsname{Содержание}
%\def\conclusionname{Выводы}
%\def\bibname{Литература}

%
% Титульный лист на русском языке
%

% Название организации
\institution{~\\ МИНИСТЕРСТВО ОБРАЗОВАНИЯ И НАУКИ РОССИЙСКОЙ ФЕДЕРАЦИИ\\ «Нижегородский государственный университет
	им. Н.\,И.\,Лобачевского» \\
	Радиофизический факультет \\
	Кафедра электродинамики \\ 
	Направление «Радиофизика»}
% Имя лица, допускающего к защите (зав. кафедрой)


% Имя лица, допускающего к защите (зав. кафедрой)
%\apname{ФИО зав. кафедрой}

\title{ВЫПУСКНАЯ  КВАЛИФИКАЦИОННАЯ  РАБОТА}

\topic{<<ИССЛЕДОВАНИЕ ОСОБЕННОСТЕЙ РАССЕЯНИЯ ЭЛЕКТРОМАГНИТНЫХ ВОЛН НА ФРАКТАЛЬНЫХ ОБЪЕКТАХ>>}

% Автор
%\author{ФИО автора}
% Группа
%\group{Студента группы}

% Научный руководитель
\sa      {ФИО руководителя}
\sastatus{д.~ф.-м.~н., ст.~н.~с.}

% Рецензент
\rev      {П.\,П.~Петров}
\revstatus{к.\,ф.-м.\,н., доцент}

% Город и год
\city{Нижний Новгород}
\date{\number\year}

\maketitle

%%
%% Titlepage in English
%%
%
%\setlength\thirdskip{0pt}
%
%\institution{Name of Organization}
%
%% Approved by
%\apname{Professor S.\,S.~Sidorov}
%
%\title{Diploma Thesis}
%
%% Topic
%\topic{Dummy Title}
%
%% Author
%\author{Author's Name} % Full Name
%\group{} % Study Group
%
%% Scientific Advisor
%\sa       {I.\,I.~Ivanov}
%\sastatus {Professor}
%
%% Reviewer
%\rev      {P.\,P.~Petrov}
%\revstatus{Associate Professor}
%
%% Consultant
%\con{}
%\conspec{}
%\constatus{}
%
%% City & Year
%\city{Saint Petersburg}
%\date{\number\year}
%
%\maketitle[en]

% Содержание
\tableofcontents

% Введение
\intro
%Рассеяние электромагнитных волн было предметом исследований с начала века. Традиционно рассеивающие объекты, которые являются большими по сравнению с длиной волны, могут быть точно смоделированы с помощью приближенного высокочастотного метода, такого как геометрическая оптика или физическая оптика. 
%Действительно, для определенного класса геометрий геометрическая и физическая оптика вполне подходят, в то время как геометрическая теория дифракции или физическая теория дифракции могут использоваться для уточнения такой модели, когда присутствуют ребра, разрывы кривизны или соединения материалов. Для объектов, которые малы по сравнению с длиной волны, полезны низкочастотные методы, такие как приближение Борна. К сожалению, низко- и высокочастотные методы предъявляют довольно жесткие требования к размеру, ориентации и составу рассеивающего объекта.  Для функций размера волны мы обычно прибегаем к методологии численного решения. \\
%Альтернативный подход, который в последнее время оказался достаточно мощным для приложений рассеяния - это метод конечных элементов. Поскольку этот метод является подходом с уравнением в частных производных, он получает преимущество от свойства локальности, присущего всем формулировкам в частных производных. Формула метода конечных элементов моделирует неоднородные диэлектрики и металлические поверхности довольно хорошо, в отличии от других способов. \\
%Метод является многообещающим подходом при моделировании больших композитных структур в контексте единой надежной формулировки. Однако, как и в случае с каждым методом уравнения в частных производных, схема усечения сетки играет важную роль в оценке точности и эффективности реализации. Точные схемы усечения сетки имеют недостаток, заключающийся в разрушении разреженности матрицы конечных элементов, но в большинстве случаев они также уменьшают вычислительную область. Приближенные схемы усечения основаны на использовании поглощающих граничных условий или поглощающих материалов, целью которых является подавление отражений волн обратно в вычислительную область.  В этой работе будет рассмотренно рассеяние плоской электромагнитной волны на рассеивателях, вычисленное с помощью метода конечных элементов и поглощающих граничных условий.

Видимый пучок света — это одна из форм электромагнитного излучения, как радиоволны и рентгеновские лучи, гамма-кванты. Подобно тому, как мы умеем управлять радиоизлучением с помощью антенн, мы также можем изменять поведение пучка света других диапазонов. Все, что видит наш глаз, — это свет, «испущенный», или, точнее, рассеянный или отраженный предметами. Качественное чистое стекло практически не рассеивает и не отражает свет — поэтому нам не составляет труда смотреть в окно, не обращая внимания на стеклянную преграду. Увидеть же предмет можно благодаря тому, что он искажает ход лучей, которое его окружает. Препятствовать этому могло бы устройство, которое «восстанавливает» световое поле и словно бы заставляет лучи света огибать предмет. Для этого нам необходимы материалы, позволяющие идеально контролировать распространение света - метаматериалы, которые способны управлять показателем преломления.
Необычные свойства метаматериалов в 1967 году предсказал советский физик Виктор Георгиевич Веселаго (сотрудник Института общей физики имени Александра Михайловича Прохорова РАН, Москва). Он показал, что материалы могут обладать необычными оптическими свойствами, как отрицательный показатель преломления и обратный эффект Доплера.
Однако лишь в 2000 году физикам впервые удалось доказать, что среды с отрицательным коэффициентом преломления действительно существуют. В отличие от классических материалов, свойства которых определяются в основном веществом, из которого они состоят, свойства метаматериалов определяются их геометрией. Чтобы добиться отрицательного преломления, необходимо, чтобы отрицательными были сразу два свойства материала — диэлектрическая проницаемость и магнитная восприимчивость.
Однако такой материал работает лишь в очень узком диапазоне длин волн, который напрямую определяется размерами и формой антенн. Чем меньше размеры антенны, тем меньше и длина волны, для которой среда имеет отрицательный коэффициент преломления. 
Фрактальные антенны – относительно новый класс электрически малых антенн, принципиально отличающийся своей геометрией. Главное отличие фрактальных геометрических форм – их дробная размерность, что внешне проявляется в рекурсивном повторении в возрастающем либо уменьшаемом масштабах исходных детерминированных или случайных шаблонов. Фрактальные технологии получили распространение при формировании средств фильтрации сигналов, синтезе трехмерных компьютерных моделей природных ландшафтов, сжатии изображений.
Первые публикации по электродинамике фрактальных структур относятся к 80-м годам прошлого века, в которых упоминается работа ученых Университета штата Пенсильвания Я.Кима
и Д.Джаггарда (Y.Kim and D.L.Jaggard). Первенство в теоретических исследованиях возможности применения фрактальных форм для формирования многополосных по частоте антенн приписывают ученому Технологического университета Каталонии К.Пуенте (C.Puente). Начало же практическому применению фрактальных антенн в 1995 году положил, как принято считать, американский инженер Натан Коэн
(N.Cohen). Возможности фракталов, такие как экономия пространства и эффективное использование всего ограниченного объема, обеспечивают явное преимущество интегрированных фрактальных антенн по сравнению с теми, которые созданы на основании Евклидовой геометрии. 
Чтобы сделать облучаемый антенной объект невидимым, метаматериал должен иметь внедрение структуры меньше длины волны. Например, для зеленого света с длиной волны 500 нм, внедренные структуры длиной около 50 нм. Чтобы произвольно искривлять путь светового луча, нам необходимо модифицировать отдельные атомы. Отрицательные значения диэлектрической и магнитной проницаемостей среды получаются в том случае, когда электроны в материале движутся в направлении, противоположном по отношению к силам, создаваемым электрическим и магнитным полями. Ключ к такого рода отрицательной реакции — использование резонансной характеристики среды, выбор колебаний со специфической частотой. 
Цель работы - исследовать такие интегральные характеристики рассеивателей при рассеянии на них плоской волны как сечение рассеяния и сечение полглощения, в зависимости от сложности геометрии объекта, материлов, состовляющих элементы, и частоты падающего излучения. 
В начале рассмотрим рассеяние плоской электромагнитной волны на рассеивателях в различных диапазонах частот для шариков, состоящих из серебра, меди и золота. Для этого представим шарики ввиде сфер для трех случаев, для одиночной сферы, для 5 сфер и для 17 сфер. Решим задачу дифракции одиночной сферы используя метод конечных элементов, и применим тот же подход для остальных случаев. А так же найдем рассеянние на цилиндре численным и точным методом решения, чтобы убедиться в правильности получаемых данных. Протестируем полученные численные схемы и проанализируем полученные результаты.
\clearpage

% Глава 1
\section{Постановка задачи и основные соотношения}

\section{Описание метода решения задачи}
Для решения поставленной задачи, нам будет необходимо рассчитать рассеяние на цилиндре численным и точным методом решения, чтобы удостовериться в правильности полученных данных. А затем, с помощью метода конечных элементов произвести вычисления полей с учетом граничных условий вблизи поверхностей объектов.
\\ 
Рассчеты для цилиндра будем производить в простейшем двумерном случае.
\\
В начале запишем уравнение Гельмгольца: \\
\begin{center}
	$ ΔH_{z} + k^2H_{z} = 0 $ 
\end{center}\\
Уравнение Максвелла: \\
\begin{center}
	$ rot\vec{H} =\frac{4\pi}{c}\vec{j} + \frac{1}{c}\frac{\partial \vec{D}}{\partial t} $ , где
\end{center}\\
\begin{center}
	$ \vec{H} = \vec{r}_{0}H_{0}cos(\omega t - kz) $, и $ \varepsilon = 1 $ \\
\end{center}
получаем : \begin{center}
	$ rot\vec{H} = ik_{0}\vec{H} $
\end{center}
Отсюда выражаем  \vec{E}: \\
\begin{center}
	$\vec{E} = - \frac{i}{k_{0}} rot\vec{H} = - \frac{i}{k_{0}}(\vec{x_{0}}\frac{\partial \vec{H_{z}}}{\partial y} - \vec{y_{0}}\frac{\partial \vec{H_{z}}}{\partial x}) 
	$
\end{center}

\begin{flushleft}
	А так же для $ E_{x} и E_{y} $ соответственно: \\
\end{flushleft}
\begin{center}
	$ E_{x} = - 
	\frac{i}{k_{0}}
	\frac{\partial\vec{H_{z}}}{\partial y} \\
	E_{y} = \frac{i}{k_{0}}
	\frac{\partial\vec{H_{z}}}{\partial x} \\ $
\end{center} \\
\begin{flushleft}
	Для условной поверхности S :
\end{flushleft} \\
\begin{center}
	$ [\vec{n}, \vec{x_{0}}E_{x} + \vec{y_{0}}E_{y}] = 0 $
\end{center}\\

\begin{flushleft}
	Таким образом получим интеграл:
\end{flushleft} \\

\begin{center}
	$ \int\limits_{V}^{} \Delta H_{z} \cdot wdv = - \int\limits_{V}^{}(\nabla H_{z}, \nabla w)dv +  \int\limits_{V}^{}(\nabla H_{z} \cdot w)dv
	$ \\
\end{center}

\begin{flushleft}
	Так как один из интегралов правой части: 
\end{flushleft}
\begin{center}
	$  \int\limits_{V}^{}(\nabla H_{z} \cdot w)dv $
\end{center} \\
\begin{flushleft}
	Соответственно равен: 
\end{flushleft}\\
\begin{center}
	$\int\limits_{V}^{}(\nabla H_{z} \cdot w)dv = \oint\limits_{S}^{} \nabla H_{z}  wds \cdot \vec{n} = 0 $ \\
\end{center}
\\
\begin{flushleft}
	То граничные условия в нашей задаче задавать нет необходимости.
	Поэтому получим:
\end{flushleft} \\

\begin{center}
	$ \nabla H_{z} = \vec{x}_{0} \frac{\partial\vec{H_{z}}}{\partial x} - \vec{y}_{0}
	\frac{\partial\vec{H_{z}}}{\partial y} $
	\\
\end{center}
\begin{flushleft}
	Cоответственно используя нормали $n_{x}$ и $n_{y}$  запишем искомое уравнение:
\end{flushleft} \\
\begin{center}
	$ \vec{n}\nabla H_{z} = \frac{\partial\vec{H_{z}}}{\partial n} = 
	n_{x}\frac{\partial\vec{H_{z}}}{\partial x} +
	n_{y}\frac{\partial\vec{H_{z}}}{\partial y} $
\end{center}

 \\
\begin{flushleft}
	Теперь решим ту же самую задачу, но уже точным методом. Для этого в начале найдем уравнения, которые нам понадобятся для решений:
\end{flushleft}\\

\begin{center}
	$ \vec{H} = \vec{r_{0}}H_{z}
\vec{H_{z}} = 1.e^{-ik_{0}x} = e^{-ik_{0}\rho cos\varphi} = 
\sum\limits_{m=-\infty}^{\infty} J_{m}(k_{0}\rho)e^{-im(\varphi + \frac{\pi}{2})} =
\sum\limits_{m=-\infty}^{\infty} (-i)^{m}J_{m}(k_{0}\rho)e^{-im\varphi}
$
\end{center}\\

\begin{flushleft}
	Запишем уравнение Гельмгольца и оператор Лапласа в следующем виде:
\end{flushleft} \\
\begin{center}
	$ 
	\Delta_{\perp}H_{z} + k_{0}H_{z} = 0, \qquad
	\Delta_{\perp} = \frac{1}{\rho}
	\frac{\partial}{\partial \rho} 
	(\rho \frac{\partial}{\partial \rho}) + 
	\frac{1}{\rho^{2}}
	\frac{\partial^{2}}{\partial \varphi^{2}} 
	$
\end{center}
\\
\begin{flushleft}
	Получим:
\end{flushleft} \\
$$ \frac{\partial^{2}}{\partial \rho^{2}}H_{z}  +
\frac{1}{\rho}\frac{\partial}{\partial \rho}H_{z} +
\frac{1}{\rho^{2}}\frac{\partial^{2}}{\partial \varphi^{2}}H_{z}+
k_{o}^{2}H_{z} = 0 $$
\\
\begin{flushleft}\\
	
	Представим $ H_{z} $ как некую $ R(\rho) $ и $ \Phi(\varphi) $:\quad
	$ H_{z} = R(\rho)\Phi(\varphi) $, \qquad тогда : \\
\end{flushleft}

$$
&& \Phi^{\prime\prime}(\varphi) + m^2\Phi(\varphi) = 0 \quad -> \Phi(\varphi) = e^{-im\varphi} \\
&& R^{\prime\prime}(\rho) + \frac{1}{\rho} R^{\prime} + 
(k_{0}^{2} - \frac{m^{2}}{\rho^{2}})R = 0
$$ 
\\
\begin{flushleft}
	Отсюда получим, что :
\end{flushleft} \\
$$
R(\rho) = A_{1}H_{m}^{(1)}(k_{0}\rho) + A_{2}H_{m}^{(2)}(k_{0}\rho)
$$
\\
Однако \quad  $ A_{1} = 0 $, \quad так как
$\quad \lim\limits_{\rho -> \infty}\sqrt{\rho}
(\frac{\partial H_{m}^{(2)}}{\partial \varphi \rho} + ik_{0}H_{m}^{(1)} )
$ \\

\begin{flushleft}
	В таком случае у нас остается: \\
\end{flushleft}

$$
H_{z} = \sum_{m=-\infty}^{\infty}A_{2m}H_{m}^{(2)}(k_{0}\rho)e^{-im\varphi}
$$
\\
\begin{flushleft}
	Граничные условия:
\end{flushleft} 
\begin{center}
	$ E_{\varphi}|_{\rho=a} = 0 $
\end{center}\\
\begin{flushleft}
	Тогда:
\end{flushleft}
$$
\vec{E}^{(s)} = - \frac{i}{k_{0}\varepsilon} rot\vec{H}^{(s)} = - \frac{i}{k_{0}} \frac{1}{\rho}
\begin{vmatrix}
\\\vec{\rho}_{0}& \;\rho \vec{\varphi}_{0}& \; \vec{z}_{0} \\
\frac{\partial }{\partial \rho}& \; 
\frac{\partial }{\partial \varphi}&\; 
\frac{\partial }{\partial z} \\
H_{\rho}&\; \rho H_{\varphi}&\; H_{z}^{(s)}
\end{vmatrix}
= \frac{i}{k_{0}} \frac{1}{\rho}(\vec{\rho_{0}} 
\frac{\partial H_{z}}{\partial \rho} - \rho\vec{\varphi}_{0}
\frac{\partial H_{z}^{(s)}}{\partial \rho}})
$$\\
$$
E_{\varphi}^{(s)} = \frac{i}{k_{0}}
\frac{\partial }{\partial \rho} \sum_{m=-\infty}^{\infty} A_{2m}H_{m}^{(2)}(k_{0}\rho)e^{-im\varphi} =
\frac{i}{k_{0}} \sum_{m=-\infty}^{\infty}A_{2m} 
\frac{\partial }{\partial \rho}
(H_{m}^{(2)}(k_{0}\rho))e^{-im\varphi}
= \\$$
\begin{center}
$$
i\sum_{m=-\infty}^{\infty}A_{2m}H_{m}^{(2)}^{\prime}(k_{0}\rho)e^{-im\varphi}
$$
\end{center}

\begin{flushleft}
Исходя из граничных условий для падающей и отраженной волны:
\end{flushleft} \\
$$
E_{\varphi}|_{\rho=a} = (E_{\varphi}^{(i)} + E_{\varphi}^{(s)})|_{\rho=a} = \sum_{m=-\infty}^{\infty}
((-i)^{m}J_{m}^{\prime}(k_{0}\rho) + A_{2m}H_{m}^{(2)}^{\prime}(k_{0}\rho))e^{-im\varphi}|_{\rho=a} = 0 \\
$$
\begin{flushleft}
Получим искомое выражение:
\end{flushleft}
$$
(-i)^{m}J_{m}^{\prime}(k_{0}a) + A_{2m}H_{m}^{(2)}^{\prime}(k_{0}a) = 0\\ $$
\begin{center}
$$
A_{2m} = (-i)^{m} \frac{J_{m}^{\prime}(k_{0}a)}
{H_{m}^{(2)}^{\prime}(k_{0}a)}
$$\\
\end{center}
\newpage
Теперь нам необходимо записать уравнение, с помощью которого мы и будем производить расчеты полей. Для этого обратимся к рис.1.
\\
\begin{figure}[h]
	\centering
	\includegraphics[width=0.6\linewidth]{tes2}
	\caption{}
	\label{fig:fr}
\end{figure}
\\
Вычислительная область ($ V_{0} $) ограничена поверхностью ($ S_{0} $) и может содержать различные рассеивающие объекты, такие как неоднородные диэлектрики ($ V_{d} $), металлические тела ($ S_{2} $) и резистивные или импедансные листы ($ S_{k} $).\\
Метод конечных элементов позволяет легко моделировать рассеиватели произвольной формы и неоднородные области, не предъявляя жестких требований к их форме, размеру и составу. Так же стоит отметить, что на протяжении всей работы будет использоваться гармоническая зависимость $ e^{j \omega t} $, где $ j = \sqrt{-1} $. \\
Ввиду того, что наш интерес ограничен рассеянием, плотности электрического и магнитного тока будут равны нулю.
С учетом граничных условий на рассеивателе (ях), следует рассмотреть его средневзвешенное значение, дающее так называемую слабую форму векторного волнового уравнения. Для векторной функции $ \vec{E^{'}} $ (1):\\
\begin{center}
	$ 	\iiint\limits_{V_{0}}^{} \left[ \nabla \times \left( \frac{1}{\mu_{r}}\nabla \times \vec{E}\right)\cdot \vec{E^{'}} - k_{0}^{2} \epsilon_{r}\vec{E} \cdot \vec{E^{'}} \right]d\upsilon = 0 $
\end{center}
Однако уравнение должно выполняться для каждого из малых объемов, а не в каждой точке $ V_{0} $. Из-за двойного завитка, прямое численное решение уравнения (1) требует расширения $ \vec{E} $ с помощью базисных функций $ S_{0} $ более высокого порядка, и мы также можем получить асимметричную систему.Чтобы избежать этих трудностей и облегчить соблюдение граничных условий, традиционный подход состоит в том, чтобы использовать первую векторную идентичность Грина и переписать формулу: \\
\begin{center}
	$ 	\iiint\limits_{V_{0}}^{} \left[ \frac{1}{\mu_{r}}(\nabla \times \vec{E}) \cdot (\nabla \times \vec{E^{'}}) - k_{0}^{2} \epsilon_{r}\vec{E} \cdot \vec{E^{'}} \right]d\upsilon - jk_{0}Z_{0} \oiint (\hat{n} \times \vec{H}) \cdot \vec{E^{'}}ds = 0,$
\end{center}\\
где $ \vec{H} $ - полное магнитное поле, удовлетворяющее уравнению Максвелла $ \nabla \times \vec{E} - jk_{0}Z_{0}\vec{H}  $.\\
Уравнение (№) является слабой формой векторного волнового уравнения. Мы можем решить его численно для $ \vec{E} $, дискретизировав объем $ V_{0} $ и расширив $ \vec{E} $ подходящим, скажем, линейным расширением внутри каждого из подобъемов $V_{e}$. \\
Также необходимо обеспечить соблюдение всех необходимых граничных условий в пределах $V_{0}$ и на $S_{0}$, что подразумевает исключение $\vec{H}$, связав его с $\vec{E}$. \\
Учитывая, что наш интерес заключается в получении рассеяния тела, освещаемого плоской волной, $S_{0}$ служит только в качестве искусственной поверхности для завершения бесконечной вычислительной области. В самом деле, если $S_{0}$ находится далеко от рассеивателя, мы можем тогда вызвать условие излучения Зоммерфельда (2): \\
\begin{center}
	$ jk_{0}Z_{0}\hat{r} \times \vec{H}^{scat} = jk_{0}\hat{r} \times \hat{r} \times \vec{E}^{scat} $ \\
	\begin{center}
связав $ \vec{H} $ и $ \vec{E} $, где $ \hat{r} $ - нормаль к сферической поверхности $ S_{0} $ и \\
		$ \vec{H}^{scat} = \vec{H} - \vec{H}^{inc},\quad \vec{E}^{scat} = \vec{E} - \vec{E}^{inc}$
	\end{center}\\
Обозначим рассеянные поля, связанные с возбуждением плоской волны ($ \vec{E}^{inc} $, $ \vec{H}^{inc} $).
\end{center}
Очевидно, что требование, чтобы $ S_{0} $ располагалась далеко от рассеивателя, значительно увеличивает вычислительный объем, что приводит к непрактично большим системам при дискретизации уравнения. Это побудило к использованию поглощающих граничных условий более высокого порядка, которые можно наносить на поверхность, находящуюся в ближней зоне рассеивателя, без существенного снижения точности решения.Их целью является устранение обратных отражений от $ S_{0} $. \\
Они обеспечивают приблизительную связь между $\vec{E}$ и $\vec{H}$ на поверхности $ S_{0} $, которую мы получаем, предполагая расширение поля в обратных степенях r, радиальное расстояние от центра $S_{0}$. Если поглощающие граничные условия аннулируют первые (2m + 1) обратные степени r, то их называют поглощающие граничные условия m-го порядка.
Поглощающие граничные условия нулевого порядка являются условием излучения Зоммерфельда, приведенным в формуле (2), а их вектор второго порядка имеет вид(3):\\
\begin{center}
	$ -jk_{0}Z_{0}\hat{n} \times \vec{H}^{scat} = jk_{0}\vec{E}_{t}^{scat} + \beta \nabla \times [\hat{n}(\nabla \times \vec{E}^{scat})_{n}] + \beta\nabla_{t} (\nabla \cdot \vec{E}_{t}^{scat})$.
\end{center} \\
В этом уравнении $ \hat{n} $ обозначает внешнюю нормаль к $S_{0}$, $\beta = 1/{2[jk_{0} + (1/r)]}$, а нижние индексы t и n обозначают тангенциальную и нормальную компоненты для $ S_{0} $ соответственно. 
Поглощающие граничные условия (уравнение (3)) были получены для сферической поверхности $ S_{0} $, однако, они хорошо работают, когда $ S_{0} $ является кусочно-плоской, чтобы она лучше соответствовала поверхности рассеивателя.
В этом случае $ \beta $ сводится к $ \beta = 1/(2jk_{0}) $.\\
Точно так же мы можем обобщить уравнение (2) для кусочно-плоских поверхностей, положив $ \hat{r} \rightarrow \hat{n} $.\\
Так же отметим, что, поскольку данные поглощающие граничные условия связывают рассеянные поля, лучше всего переписать формулу (1) в терминах рассеянных полей. Делая так, мы получаем (4): \\
\begin{center}
	$ \iiint\limits_{V_{0}}^{} \left[ \frac{1}{\mu_{r}}(\nabla \times \vec{E}^{scat}) \cdot (\nabla \times \vec{E^{'}}) - k_{0}^{2} \epsilon_{r}\vec{E}^{scat} \cdot \vec{E^{'}} \right]d\upsilon + 
	\iint\limits_{S_{0}}^{}  P(\vec{E}^{scat}) \cdot \vec{E^{'}}ds + 
	\iiint\limits_{V_{d}}^{} \left[ \frac{1}{\mu_{r}}(\nabla \times \vec{E}^{inc}) \cdot (\nabla \times \vec{E^{'}}) - k_{0}^{2} \epsilon_{r}\vec{E}^{inc} \cdot \vec{E^{'}} \right]d\upsilon + 
	jk_{0}Z_{0}\iint\limits_{S_{d}}^{} \frac{1}{\mu_{r}}(\hat{n} \times \vec{H}^{inc}) \cdot \vec{E^{'}}ds = 0, $
\end{center}\\
в котором $P(\vec{E^{scat}})$ равно представлению в правой части уравнения (2), $ V_{d} $ - это объем, занимаемый диэлектриками, а $ S_{d} $ - это поверхность между диэлектрическими интерфейсами. Мы вывели последние два интеграла в формуле (4) снова вызвав первое векторное тождество Грина и отметим, что $ \vec{E^{inc}} $ удовлетворяет волновому уравнению вектора свободного пространства. Очевидно, что (4) относится только к неизвестным $ \vec{E^{scat}} $, и мы можем приступить к его решению.
\section{Тестирование численной схемы}
\begin{flushleft}
	Приступим к созданию трехмерной модели в программе FreeFem на основе полученных нами выше выражений.  Для этого выберем наиболее подходящий программный способ реализации 3D объектов, а так же сразу определим, что наш объект будет в виде сферы(у которой тут же зададим радиус и другие параметры). 
	\\
	Возьмем сферическую область, в которой будут рассеиваться волны и заключим в нее наш объект - сферу более маленького радиуса заполненного диэлектриком.
\end{flushleft} \\
\begin{flushleft}
	Определим параметры характеризующие наше поле,а именно:
\end{flushleft} \\
\begin{flushleft}
	$ \lambda = 10^{-6} $ - длина волны в вакууме в метрах \\
\end{flushleft}
\begin{flushleft}
	$ c = 299792458 $ - скорость света в вакууме \\
\end{flushleft}
\begin{flushleft}
	$ \mu = 4\pi 10^{-7} $ - магнитная проницаемость в свободном пространстве \\
\end{flushleft}
\begin{flushleft}
	$ \varepsilon = \frac{\mu}{c^{2}} $ - диэлектрическая проницаемость в свободном пространстве \\
\end{flushleft}
\begin{flushleft}
	$ Z = \sqrt{\frac{\mu}{\varepsilon}} $ - импеданс свободного пространства \\
\end{flushleft}
\begin{flushleft}
	$ n = 2 $ - показатель преломления сферы \\
\end{flushleft}
\begin{flushleft}
	Радиус внутренней сферы (нашего объекта) зададим в 10 раз меньше длины волны принимаемой для расчетов ($ \lambda = 1 $), а радиус внешней сферы в 4 раза больше. Таким образом получим систему состоящую из двух сфер разного радиуса (одна внутри другой), взаимодействие между которыми опиcываются с помощью уравнений полученных ранее. Однако, прежде чем мы это сделаем, определим в программе область расчетов через радиус внутренней сферы, для случаев когда он больше или меньше радиуса применяемого для вычислений полей в ближней и дальней зоне. Необходимость этого заключается в том, что уравнение должно выполняться для каждого из малых объемов, а не в каждой точке, как было сказано ранее.
\end{flushleft}
\\ 
\begin{flushleft}
	Имея уравнения Лапласа:
\end{flushleft}
\\
\begin{center}
	$ \Delta \vec{E} + k^2\vec{E}=0 $
\end{center}
Опрелелим k для случаев:
\begin{center}
	k = 
	\begin{cases}
		1, & r > a\\
		n, & r < a ,\qquad $ где а - радиус малой сферы $
	\end{cases}
\end{center}
Далее добавим цикл, который ввиду гармонической зависимости от времени будет производить перерасчет данных и выводить результаты на экран, благодаря чему получится наглядная анимированная картина поля. А так же добавим возможность сохранения полученных данных для Matlab.\\
После того, как мы посчитали дифракцию на сфере, перейдем непосредственно к расчету сечения рассеяния и сечения поглощения.\\
Если на рассеивающий элемент падает волна с интенсивностью I(под интенсивностью понимается поток энергии через единичную площадку), то полная рассеянная мощность S будет пропорциональна I. Коэффициент пропорциональности между этими величинами $ \sigma_s $ называется полным сечением рассеяния и имеет размерность площади: \\
\begin{center}
	$ \sigma_s = \frac{S}{I} $
\end{center}
Так же введем понятие дифференциального сечения рассеяния
 $ \sigma_d(\theta,\varphi) $. Пусть $ dS(\theta, \varphi) $ - полная мощность, рассеянная в пределах телесного угла d$ \Omega $ в направлении $ (\theta,\varphi) $, тогда:
 \begin{center}
 	$ \sigma_d(\theta,\varphi) = \lim\limits_{d\Omega\rightarrow\infty} \frac{dS(\theta, \varphi)}{Id\Omega}. $s
 \end{center}
Приминительно к нашей задаче, это выражение примет вид:
\begin{equation}
\sigma_d(\theta,\varphi) = \lim\limits_{R\rightarrow\infty} \frac{R^2 S_r(\theta, \varphi)}{S_i}.
\end{equation}
Здесь вектор Пойнтинга падающей волны
\begin{equation}
{\vec S}_i = \frac{1}{2} \left[{\vec E}_i \times {\vec H}^*_i\right],
\end{equation}
радиальная компонента вектора Пойнтинга рассеянных волн
\begin{equation}
{S}_r = \frac{1}{2} \left[{\vec E}_s \times {\vec H}^*_s\right] \cdot {\vec \rho}_0
\end{equation}

В нашем случае при расчётах необходимо будет взять значения вектора Пойнтинга рассеянных волн на поверхности с фиксированным значением $R$ (приемлемо, если $R=3\lambda$, т.к. внешняя граница удалена на расстояние $4\lambda$).

Наиболее простой способ расчёта $S_r$ через компоненты в декартовой системе координат:
\begin{equation}
{S}_r = S_x \cos(\varphi)\sin(\theta) + S_y \sin(\varphi)\sin(\theta) + S_z \cos(\theta).
\end{equation}
Здесь
\begin{eqnarray}
& S_x = \frac{1}{2} \left(E_y H_z^* - E_z H_y^*\right),\nonumber\\
& S_y = \frac{1}{2} \left(E_z H_x^* - E_x H_z^*\right),\nonumber\\ & S_z = \frac{1}{2} \left(E_x H_y^* - E_y H_x^*\right).
\end{eqnarray}
Предполагается, что поля относятся к рассеянному полю (индеркс $s$ опущен).
Компоненты магнитного поля находим из уравнения Максвелла
\begin{equation}
{\vec H} = \frac{i}{k_0} {\rm rot} \vec{E}.
\end{equation}

Полное сечение рассеяния находится как интеграл по полному телесному углу:
\begin{equation}
\sigma_s = \int_{4\pi} \sigma_d d\Omega.
\end{equation}

Сечение поглощения определяется как отношение полной мощности, теряемой первичной волной и преобразующейся в тепло в данной локальной области, к плотности потока энергии(интенсивности) в первичной волне, его находим как:
\begin{equation}
\sigma_a = \left(\int_V \frac{1}{2}\omega\varepsilon_0\varepsilon''|{\vec E}|^2 d V\right)/S_i.
\end{equation}
В случае, если падающее поле имеет единичную амплитуду ($|E_i|=1$):
\begin{equation}
\sigma_a = \int_V k \varepsilon''|{\vec E}|^2 d V.
\end{equation}
Здесь интегрирование проводится по области занятой диэлектрическим рассеивателем.

Вычисление действительной и мнимой частей диэлектрической проницаемости на основе коэффициента преломления будет выглядеть как:
\begin{equation}
\varepsilon' - i \varepsilon'' = (n'-i n'')^2.
\end{equation}
Отсюда действительная $ \varepsilon' $ и мнимая $ \varepsilon'' $ часть будут соответственно равны:
\begin{equation}
\varepsilon' = (n')^2 - (n'')^2,\quad \varepsilon'' = 2 n'n''.
\end{equation}

\section{Результаты численных расчётов}



% Глава 2
%\input{2}

% Заключение
\conclusion
В настоящей работе рассмотрены такие интегральные характеристики рассеивателей при рассеянии на них плоской волны как сечение рассеяния и сечение поглощения, при разной геометрии объекта, материалов, составляющих элементы, и частоты падающего излучения. Получены выражения для рассеяния плоской электромагнитной волны на одиночном цилиндре численным и точным методом. Рассмотрена задача дифракции одиночной сферы используя метод конечных элементов. Были протестированы полученные численные схемы с помощью программы FreeFem. Установлены зависимости частот от длин волн при рассеянии на разных типах и колличествах сфер. А именно, получены графики для сечения поглощения и сечения рассеяния на одиночном медном,серебряном,золотом шаре. Их первой итерации состоящей из 5 шаров, а так же для второй инетарции состоящей из 17 шаров. Показано, что увеличение сложности рассеивателя приводит к увеличению сечения рассеяния для всех сред заполнения и всех длин волн, за исключением небольшого интервала в оптической области частот для серебряных рассеивателей в котором семнадцать шаров имеют меньшее сечение рассеяния, чем рассеиватель из пяти шаров.
В тоже время, сечения поглощения для всех материалов в инфракрасной области частот возрастает с увеличением числа элементов рассеивателя, а в оптической области сечение поглощения рассеивателя из семнадцати шаров уменьшается по сравнению с сечением поглощения рассеивателя из пяти шаров.
При этом установлено, что сечение поглощения и сечение рассеяния имеют наименьшие значения практически на всех длинах волн для рассеивателей, составленных из серебра. 
Показано, что для серебра и золота, локальный минимум зависимости сечения поглощения от длины волны смещается в сторону больших длин волн с увеличением сложности рассеивателя.



% Список литературы
\printbibliography[heading=bibintoc]
\begin{thebibliography}{99}	
	\bibitem{b4} Torres J.\,P., Torner L.  Twisted photons. Singapore: Wiley,  2011. 288 p.
	
	\bibitem{b2}  Wang J., Yang J.-Y., Fazal I.\,M., et al. // Nat. Photon. 2012. {V.~6}, No.~7. P.\,488--496. doi:10.1038/nphoton.2012.138
	
	\bibitem{b3} Hiesmayr B.\,C., de Dood M.\,J.\,A., Loffler W. // Phys. Rev. Let. 2016. V.~116, No.~7. Art.~no.~073601. doi:10.1103/PhysRevLett.116.073601
	
	\bibitem{b10} Hu L.-X., Yu T.-P., Lu Y., et al. // Plasma Phys. Control. Fusion. 2018. {V. 61}, No.~2. Art.~no.~025009. doi:10.1088/1361-6587/aaefb6
	
	\bibitem{b11} Baumann C., Pukhov A. // Phys. Plasmas. 2018. V. 25, No.~8. Art.~no.~083114. doi:10.1063/1.5044617
	
	\bibitem{Gusynin2007}
	Gusynin V.\,P., Sharapov S.\,G., Carbotte J.\,P. // J. Phys.: Condens. Matter. 2007. V.~19, No.~2. Art.~no.~026222. doi:10.1088/0953-8984/19/2/026222
	
	\bibitem{Kondratev1999} Kondrat'ev I.\,G., Kudrin A.\,V., Zaboronkova T.\,M. Electrodynamics of density ducts in magnetized plasmas. Amsterdam: Gordon and Breach, 1999. 288 p.
	
	\bibitem{b6}
	Вайнштейн Л.\,А. {Электромагнитные волны}. Москва: Радио и связь, 1988. 440 c.
	
	%\bibitem{Shevchenko1971} Shevchenko V.~V. {Continuous Transitions in Open Waveguides}. Boulder: Golem Press,  1971. 176 p.
	\bibitem{Shevchenko1971} Шевченко В.~B. Плавные переходы в открытых волноводах. Москва: Наука,
	1969. 192 с.
	
	\bibitem{Eskin2017} Es'kin V.\,A., Kudrin A.\,V. // Proc. PIERS 2017. 2017. P.~843. doi:10.1109/PIERS.2017.8261860
	
\end{thebibliography}

% Приложения
%\appendix
%
\chapter{Код программы для расчета рассеянного поля при рассеянии плоской волны на идеально-проводящем цилиндре}


\begin{lstlisting}[language=C]
real k =4.;
real w = 4.;
real t = 0.;

real a = 20.;
border a0(t=0., 2.*pi){x=5.*cos(t); y=5.*sin(t);}
border s0(t=0., 2.*pi){x=cos(t); y=sin(t);}

int n = 30;
mesh Th = buildmesh(a0(2*n) + s0(-20));

plot(Th,wait=1);

fespace Vh(Th, P2);
Vh<complex> Ex, ex;
Vh<complex> Ey, ey;
Vh<complex> Hz, hz;
t = 0;
Vh<complex> Hz0 = exp(1i* w* t - 1i* k * x);

solve GelmhotzEquationForH(Hz, hz)
= int2d(Th)
(
- dx(Hz)* dx(hz)
- dy(Hz)* dy(hz)
+ k^2 * Hz* hz 
)
+ int1d(Th,s0)
(
- dx(Hz0)* N.x* hz - dy(Hz0)* N.y* hz
)
+ int1d(Th,a0)
(
-1i*k*Hz*hz
)
;

real dt = 2.* pi/ w / 20;

GelmhotzEquationForH;

Vh realHz, realEx, realEy;
Vh<complex> HzInTime = 0.;
for (int i=1; i < 100; i++)
{
t = dt * i;
cout<< "Time is" << t << endl;  


HzInTime = Hz* exp(1i * w * t) + 0*exp(1i* w* t - 1i* k * x);
Ex = - 1i/ k * dy(HzInTime);
Ey =   1i/ k * dx(HzInTime);

realHz = real(HzInTime);
plot(realHz, wait=0, value=1, fill = 1);

realEx = real(Ex);
realEy = real(Ey);

plot(realEx, wait=0, value=1, fill=1, WindowIndex=1);
plot(realEy, wait=0, value=1, fill=1, WindowIndex=2);
}
\end{lstlisting}

\end{document}
























%
%
%\documentclass[bachelor,subf,12pt,notitlepage]{disser}
%
%%\usepackage{pscyr}
%\usepackage[
%a4paper, mag=1000, includefoot,
%left=3cm, right=1cm, top=2cm, bottom=1.5cm, headsep=1cm, footskip=1cm
%]{geometry}
%
%\usepackage[T2A]{fontenc}
%\usepackage[cp1251]{inputenc}
%\usepackage[english,russian]{babel}
%\usepackage{multirow}
%\usepackage{longtable}
%
%\ifpdf\usepackage[pdftex]{graphicx}\else\usepackage{graphicx}\fi
%
%\usepackage{comment}
%
%\ifpdf\usepackage{epstopdf}\usepackage{pdfpages}\fi
%
%\graphicspath{{fig/}}
%\renewcommand{\thechapterfont}{\normalsize\bfseries} % Номер главы полужирным
%\renewcommand{\prethechapter}{} % Убираем слово "глава"
%\renewcommand{\postthechapter}{.~} % ставим точку и пробел после номер
%\renewcommand{\appendixfont}{\normalsize\bfseries}
%\renewcommand{\chapterfont}{\normalsize\bfseries}
%\renewcommand{\sectionfont}{\normalsize\bfseries}
%\renewcommand{\subsectionfont}{\normalsize\bfseries}
%\renewcommand{\subsubsectionfont}{\normalsize\bfseries}
%\renewcommand{\tocprethechapter}{} % в оглавлении убираем слово "Глава"
%\renewcommand{\theappendixalign}{\hfill} % выключка вправо для слова "приложение" в приложениии
%\renewcommand{\theappendix}{\arabic{chapter}} % заменяем нумерацию приложений на цифры
%\renewcommand{\pretheappendix}{\protect{ПРИЛОЖЕНИЕ}~} % Меняем регистр слова "Приложение"
%\renewcommand{\tocpretheappendix}{\protect{ПРИЛОЖЕНИЕ}~} % Меняем регистр слова "Приложение"
%\renewcommand{\introname}{ВВЕДЕНИЕ}
%\renewcommand{\sectionindent}{1cm}
%\renewcommand{\subsectionindent}{1cm}
%\renewcommand{\aftersection}{6pt plus .1pt}
%\renewcommand{\aftersubsection}{3pt plus .1pt}
%\renewcommand{\conclusionname}{ЗАКЛЮЧЕНИЕ}
%\renewcommand{\bibname}{БИБЛИОГРАФИЧЕСКИЙ СПИСОК}
%
%\setcounter{tocdepth}{2}
%
%\begin{document}
%	
%	\setcounter{page}{2}
%	\renewcommand{\contentsname}{ОГЛАВЛЕНИЕ}
%	
%	% ----------------------------------------------------------------
%	\tableofcontents
%	% ----------------------------------------------------------------
%	
%	\addcontentsline{toc}{section}{Введение}
%	\section*{Введение}
%	\indent
%	
%	
%	\section{Постановка задачи и основные соотношения}
%	
%	\section{Результаты численных расчётов}
%	
%	
%	\newpage
%	\addcontentsline{toc}{section}{Заключение}
%	\section*{Заключение}
%	\indent
%	% ----------------------------------------------------------------
%	
%	
%	\renewcommand{\bibname}{БИБЛИОГРАФИЧЕСКИЙ СПИСОК}
%	\bibliography{thesis}
%	\bibliographystyle{gost705}
%\end{document}
