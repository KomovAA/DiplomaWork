
\chapter{Код программы для расчета рассеянного поля при рассеянии плоской волны на идеально-проводящем цилиндре}


\begin{lstlisting}[language=C]
real k =4.;
real w = 4.;
real t = 0.;

real a = 20.;
border a0(t=0., 2.*pi){x=5.*cos(t); y=5.*sin(t);}
border s0(t=0., 2.*pi){x=cos(t); y=sin(t);}

int n = 30;
mesh Th = buildmesh(a0(2*n) + s0(-20));

plot(Th,wait=1);

fespace Vh(Th, P2);
Vh<complex> Ex, ex;
Vh<complex> Ey, ey;
Vh<complex> Hz, hz;
t = 0;
Vh<complex> Hz0 = exp(1i* w* t - 1i* k * x);

solve GelmhotzEquationForH(Hz, hz)
= int2d(Th)
(
- dx(Hz)* dx(hz)
- dy(Hz)* dy(hz)
+ k^2 * Hz* hz 
)
+ int1d(Th,s0)
(
- dx(Hz0)* N.x* hz - dy(Hz0)* N.y* hz
)
+ int1d(Th,a0)
(
-1i*k*Hz*hz
)
;

real dt = 2.* pi/ w / 20;

GelmhotzEquationForH;

Vh realHz, realEx, realEy;
Vh<complex> HzInTime = 0.;
for (int i=1; i < 100; i++)
{
t = dt * i;
cout<< "Time is" << t << endl;  


HzInTime = Hz* exp(1i * w * t) + 0*exp(1i* w* t - 1i* k * x);
Ex = - 1i/ k * dy(HzInTime);
Ey =   1i/ k * dx(HzInTime);

realHz = real(HzInTime);
plot(realHz, wait=0, value=1, fill = 1);

realEx = real(Ex);
realEy = real(Ey);

plot(realEx, wait=0, value=1, fill=1, WindowIndex=1);
plot(realEy, wait=0, value=1, fill=1, WindowIndex=2);
}
\end{lstlisting}