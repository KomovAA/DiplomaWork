\chapter{Код программы для расчета сечений рассеяния и поглощения, структуры поля в случае рассеяния плоской волны на фрактальном объекте}

\begin{lstlisting}[language=C]
verbosity=1;
load "msh3"
load "tetgen"
load "medit"
include "MeshSurface.idp"
include "ffmatlib.idp"


real c0 = 299792458;        // speed of light in vacuum
real mue0 = 4.*pi*1e-7;     // Permeability of free space
real eps0 = 1./c0^2/mue0;   // Permittivity of free space
real Z0 = sqrt(mue0/eps0);  // Impedance of free space

real    lambda0 = 10e-6;        // vacuum wavelength (in m)
complex ns    = 10.- 1i*70.;   // sphere refractive index (complex)
ns = 2.;
// ns = 1.; // test - uniform media

real lambda = 1.;        // vacuum wavelength (in 10*1e-6 m)
real radiusOfInnerSpher = lambda / 10.;

real k0 = 2.* pi / lambda;
complex k2 = k0 * ns;
real w = c0 * 2.* pi / lambda0;
real alpha=1; 	//penalty term

meshS ThHex;
real volumetet;  // use in tetg.

//real step = 0.2/radiusOfInnerSpher;
real step = lambda / 20.;
//cout << "step = "<< step << endl;

real volumetetIn = (step^3)/6.;  
//step = lambda / 2;

//////////////////////////////////////////////////////////////////////////
//////////////////// creating an inner spheres ///////////////////////////////////
////// zero iteration
meshS ThInnerSphere; //
ThInnerSphere = Sphere(radiusOfInnerSpher,    step,   11, -1);
//medit("Scatterer", ThInnerSphere,wait=1);
/////////////////////////// first iteration ///////////////////////////////////////////////////

meshS ThInnerSphere2 = Sphere(radiusOfInnerSpher/2, step/2, 13, -1);
real ddx = radiusOfInnerSpher + radiusOfInnerSpher/2;
ThInnerSphere2 = movemeshS(ThInnerSphere2,transfo=[x + ddx,y,z]);
meshS ThInnerSphere3 = Sphere(radiusOfInnerSpher/2, step/2, 14, -1);
ThInnerSphere3 = movemeshS(ThInnerSphere3,transfo=[x - ddx,y,z]);

meshS ThInnerSphere4 = Sphere(radiusOfInnerSpher/2, step/2, 15, -1);
ThInnerSphere4 = movemeshS(ThInnerSphere4,transfo=[x,y + ddx,z]);
meshS ThInnerSphere5 = Sphere(radiusOfInnerSpher/2, step/2, 16, -1);
ThInnerSphere5 = movemeshS(ThInnerSphere5,transfo=[x,y - ddx,z]);
//medit("Bounday mesh",ThInnerSphere,wait=1);


/////////////////////////////////////////////////////////////////////////////
meshS ThS =  ThInnerSphere + ThInnerSphere2 + ThInnerSphere3 + ThInnerSphere4 + ThInnerSphere5;
//medit("Bounday mesh",ThS,wait=1);

////////////////// second iteration //////////////////////////////////////////////////////////
meshS ThInnerSphere21 = Sphere(radiusOfInnerSpher/4, step/4, 17, -1);
meshS ThInnerSphere22 = Sphere(radiusOfInnerSpher/4, step/4, 18, -1);
meshS ThInnerSphere23 = Sphere(radiusOfInnerSpher/4, step/4, 19, -1);
real ddy = radiusOfInnerSpher/2. + radiusOfInnerSpher/4.;
ThInnerSphere21 = movemeshS(ThInnerSphere21,transfo=[x + ddx,y + ddy,z]);
ThInnerSphere22 = movemeshS(ThInnerSphere23,transfo=[x + ddx + ddy,y,z]);
ThInnerSphere23 = movemeshS(ThInnerSphere23,transfo=[x + ddx,y - ddy,z]);

ThS =  ThS + ThInnerSphere21 + ThInnerSphere23 + ThInnerSphere22;


meshS ThInnerSphere31 = Sphere(radiusOfInnerSpher/4, step/4, 20, -1);
meshS ThInnerSphere32 = Sphere(radiusOfInnerSpher/4, step/4, 21, -1);
meshS ThInnerSphere33 = Sphere(radiusOfInnerSpher/4, step/4, 22, -1);
ThInnerSphere31 = movemeshS(ThInnerSphere31,transfo=[x - ddx,y + ddy,z]);
ThInnerSphere32 = movemeshS(ThInnerSphere33,transfo=[x - ddx - ddy,y,z]);
ThInnerSphere33 = movemeshS(ThInnerSphere33,transfo=[x - ddx,y - ddy,z]);

ThS =  ThS + ThInnerSphere31 + ThInnerSphere33 + ThInnerSphere32;


meshS ThInnerSphere41 = Sphere(radiusOfInnerSpher/4, step/4, 23, -1);
meshS ThInnerSphere42 = Sphere(radiusOfInnerSpher/4, step/4, 24, -1);
meshS ThInnerSphere43 = Sphere(radiusOfInnerSpher/4, step/4, 25, -1);
ThInnerSphere41 = movemeshS(ThInnerSphere41,transfo=[x,y + ddx + ddy,z]);
ThInnerSphere42 = movemeshS(ThInnerSphere43,transfo=[x + ddy,y + ddx,z]);
ThInnerSphere43 = movemeshS(ThInnerSphere43,transfo=[x - ddy,y + ddx,z]);

ThS =  ThS + ThInnerSphere41 + ThInnerSphere43 + ThInnerSphere42;


meshS ThInnerSphere51 = Sphere(radiusOfInnerSpher/4, step/4, 26, -1);
meshS ThInnerSphere52 = Sphere(radiusOfInnerSpher/4, step/4, 27, -1);
meshS ThInnerSphere53 = Sphere(radiusOfInnerSpher/4, step/4, 28, -1);
ThInnerSphere51 = movemeshS(ThInnerSphere51,transfo=[x,y - ddx - ddy,z]);
ThInnerSphere52 = movemeshS(ThInnerSphere53,transfo=[x + ddy,y - ddx,z]);
ThInnerSphere53 = movemeshS(ThInnerSphere53,transfo=[x - ddy,y - ddx,z]);

ThS =  ThS + ThInnerSphere51 + ThInnerSphere53 + ThInnerSphere52;
medit("Scatterer",ThS,wait=1);

//////////////////////////////////////////////////////////////////////////////
///////////////////////////// creating an outer sphere //////////////////////////
step = lambda / 2.;
volumetet = (step^3)/6.;  
meshS ThOuterSphere; // 
real radiusOfOuterSphere = 4.* lambda;
ThOuterSphere = Sphere(radiusOfOuterSphere, step,12,1);

meshS ThVirtualSphere; // 
real radiusOfVirtualSphere = 3.* lambda;
ThVirtualSphere = Sphere(radiusOfVirtualSphere, step, 123, -1);
ThS =  ThS + ThOuterSphere + ThVirtualSphere;
medit("3td iteration",ThS,wait=1);
/////////////////////////////////////////////////////////////////////////////

real[int] domaine  = [0,0,radiusOfOuterSphere-0.1,1,volumetet,0.,0,0.,2,volumetetIn,0.,0,radiusOfVirtualSphere-0.1,3,volumetet,ddx,0,0.,4,volumetetIn,-ddx,0,0,5,volumetetIn,0,ddx,0,6,volumetetIn,0,-ddx,0,7,volumetetIn, ddx,ddy,0,8,volumetetIn,ddx,-ddy,0.,9,volumetetIn,ddx+ddy,0,0.,10,volumetetIn,-ddx,ddy,0,11,volumetetIn,-ddx,-ddy,0,12,volumetetIn,-ddx-ddy,0,0,13,volumetetIn, 0,ddx+ddy,0,14,volumetetIn, -ddy,ddx,0,15,volumetetIn, ddy,ddx,0,16,volumetetIn, 0, -ddx-ddy,0,17,volumetetIn, ddy,-ddx,0,18,volumetetIn, -ddy, -ddx,0,19,volumetetIn];
//
mesh3 Th = tetg(ThS,switch="pqaAAYYQ",nbofregions=19,regionlist=domaine);
// Tetrahelize the interior of the cube with tetgen
medit("3td iteration",Th,wait=1);

fespace Ph(Th,P03d);
fespace Vh(Th,P1);
Ph reg = region;           // function that returns region number of a point XY
Ph inReg  = sqrt(x^2.+y^2. + z^2.) < radiusOfInnerSpher;  //subdomains for inside the sratterer
Ph outReg = sqrt(x^2.+y^2. + z^2.) > radiusOfInnerSpher;  //subdomains for outside the scatterer
Vh dd = inReg;

//plot(dd,wait=1,value=1);

cout << "  centre = " << reg(0,0,0) << endl;
cout << " virtual = " << reg(0,0,radiusOfVirtualSphere-0.1) << endl;
cout << " exterieur = " << reg(0,0,radiusOfOuterSphere-0.1) << endl;

macro Grad(u) [dx(u),dy(u),dz(u)] // EOM
macro Curl(ux, uy, uz)[dy(uz)-dz(uy), dz(ux)-dx(uz), dx(uy)-dy(ux)]// EOM
macro Div(ux, uy, uz)[dx(ux) + dy(uy) + dz(uz)]// EOM
macro CrossN(ux, uy, uz)[uy*N.z-uz*N.y, uz*N.x-ux*N.z, ux*N.y-uy*N.x]// EOM
macro Curlabs(ux, uy, uz)[abs(dy(uz)-dz(uy)), abs(dz(ux)-dx(uz)), abs(dx(uy)-dy(ux))]// EOM

real t = 0.;

Vh<complex> Hz, hz;
Vh<complex> Ex1 = exp(- 1i* k0 * z);
Vh <complex> Ez,ez,Ey,ey,Ex,ex,Hy,Sx,Sy,Sz,Sr,Hx,Si,SigmD;
Vh <complex> N0;


func complex Eps() {
if (Th(x,y,z).region == Th(0.0,0.0,0.0).region || Th(x,y,z).region == Th(ddx,0.0,0.0).region || Th(x,y,z).region == Th(-ddx,0.0,0.0).region || Th(x,y,z).region == Th(0.0, ddx, 0.0).region || Th(x,y,z).region == Th(0.0,-ddx,0.0).region ||
Th(x,y,z).region == Th(ddx,ddy,0.0).region || Th(x,y,z).region == Th(ddx+ddy,0.0,0.0).region || Th(x,y,z).region == Th(ddx,-ddy,0.0).region ||
Th(x,y,z).region == Th(-ddx,ddy,0.0).region || Th(x,y,z).region == Th(-ddx-ddy,0.0,0.0).region || Th(x,y,z).region == Th(-ddx,-ddy,0.0).region ||
Th(x,y,z).region == Th(0.0,ddx+ddy,0.0).region || Th(x,y,z).region == Th(ddy,ddx,0.0).region || Th(x,y,z).region == Th(-ddy,ddx,0.0).region ||
Th(x,y,z).region == Th(0.0,-ddx-ddy,0.0).region || Th(x,y,z).region == Th(ddy,-ddx,0.0).region || Th(x,y,z).region == Th(-ddy,-ddx,0.0).region)
return ns^2;
else 
return 1.0;
}

func real InInner() {
if (Th(x,y,z).region == Th(0.0,0.0,0.0).region || Th(x,y,z).region == Th(ddx,0.0,0.0).region || Th(x,y,z).region == Th(-ddx,0.0,0.0).region || Th(x,y,z).region == Th(0.0, ddx, 0.0).region || Th(x,y,z).region == Th(0.0,-ddx,0.0).region ||
Th(x,y,z).region == Th(ddx,ddy,0.0).region || Th(x,y,z).region == Th(ddx+ddy,0.0,0.0).region || Th(x,y,z).region == Th(ddx,-ddy,0.0).region ||
Th(x,y,z).region == Th(-ddx,ddy,0.0).region || Th(x,y,z).region == Th(-ddx-ddy,0.0,0.0).region || Th(x,y,z).region == Th(-ddx,-ddy,0.0).region ||
Th(x,y,z).region == Th(0.0,ddx+ddy,0.0).region || Th(x,y,z).region == Th(ddy,ddx,0.0).region || Th(x,y,z).region == Th(-ddy,ddx,0.0).region ||
Th(x,y,z).region == Th(0.0,-ddx-ddy,0.0).region || Th(x,y,z).region == Th(ddy,-ddx,0.0).region || Th(x,y,z).region == Th(-ddy,-ddx,0.0).region)
return 1.;
else 
return 0.;
}


//////////////////// calculation of the scattered field /////////////////////  
problem MaxwellEquationsForH3D([Ex,Ey,Ez],[ex,ey,ez]) =
int3d(Th)(Curl(ex,ey,ez)'*Curl(Ex,Ey,Ez))
- int3d(Th)(k0^2 * Eps() * [ex,ey,ez]'*[Ex,Ey,Ez])
+ int2d(Th,11)( 1i*k0 * ( ez * N.x - ex * N.z) * Ex1)
+ int2d(Th,13)( 1i*k0 * ( ez * N.x - ex * N.z) * Ex1)
+ int2d(Th,14)( 1i*k0 * ( ez * N.x - ex * N.z) * Ex1)
+ int2d(Th,15)( 1i*k0 * ( ez * N.x - ex * N.z) * Ex1)
+ int2d(Th,16)( 1i*k0 * ( ez * N.x - ex * N.z) * Ex1)

+ int2d(Th,17)( 1i*k0 * ( ez * N.x - ex * N.z) * Ex1)
+ int2d(Th,18)( 1i*k0 * ( ez * N.x - ex * N.z) * Ex1)
+ int2d(Th,19)( 1i*k0 * ( ez * N.x - ex * N.z) * Ex1)
+ int2d(Th,20)( 1i*k0 * ( ez * N.x - ex * N.z) * Ex1)
+ int2d(Th,21)( 1i*k0 * ( ez * N.x - ex * N.z) * Ex1)
+ int2d(Th,22)( 1i*k0 * ( ez * N.x - ex * N.z) * Ex1)
+ int2d(Th,23)( 1i*k0 * ( ez * N.x - ex * N.z) * Ex1)
+ int2d(Th,24)( 1i*k0 * ( ez * N.x - ex * N.z) * Ex1)
+ int2d(Th,25)( 1i*k0 * ( ez * N.x - ex * N.z) * Ex1)
+ int2d(Th,26)( 1i*k0 * ( ez * N.x - ex * N.z) * Ex1)
+ int2d(Th,27)( 1i*k0 * ( ez * N.x - ex * N.z) * Ex1)
+ int2d(Th,28)( 1i*k0 * ( ez * N.x - ex * N.z) * Ex1)
+ int2d(Th,12)(-1i*k0 * ((ex * N.x + ey * N.y + ez * N.z)*(Ex * N.x + Ey * N.y + Ez * N.z) - (Ex * ex + Ey * ey + Ez * ez)))
;

MaxwellEquationsForH3D;

real R = 3. * lambda;
Vh Phi = atan(y/x);
Vh teta = acos(z / sqrt(x^2 + y^2 + z^2));
complex E = 1.;
Hx = 1i/k0 * (dy(Ez)-dz(Ey));
Hy = 1i/k0 * (dz(Ex)-dx(Ez));
Hz = 1i/k0 * (dx(Ey)-dy(Ex));
Sx = 1./2. * (Ey*conj(Hz) - Ez*conj(Hy));
Sy = 1./2. * (Ez*conj(Hx) - Ex*conj(Hz));
Sz = 1./2. * (Ex*conj(Hy) - Ey*conj(Hx));
Sr = Sx * cos(Phi) * sin(teta) + Sy * sin(Phi) * sin (teta) + Sz * cos(teta);
Vh R2 = x^2+y^2+z^2;
Si = 1.;

SigmD = R2 * Sr/Si;

complex sigmaTotal = (int2d(Th, 123)(R2 * Sr));
cout << "sigmaTotal = " << sigmaTotal << endl;

real sigmaT = (int2d(Th, 123)(R2 * real(Sr)));
cout << "sigmaT = " << sigmaT << endl;

Vh <complex> EpsI; // EpsI - epsilon imaginary

EpsI = 2*real(ns)*imag(ns);

complex sigmaAbsorption = (int3d(Th) ( InInner() *(k0 * EpsI * (Ex * conj(Ex)  +   Ey * conj(Ey) +   Ez * conj(Ez))))); 
cout << "sigmaAbsorption = " << sigmaAbsorption << endl;		 


Vh realHz, realEy, realEx, realEx1;
Vh<complex> HzInTime = 0.;
Vh<complex> Ex1InTime = 0.;
Vh<complex> E0=1.;
Ex1InTime = Ex + (1. - InInner()) * Ex1;
realEx1 = real(Ex1InTime);

cout << "k0 = " << k0 << endl;

//Ex1InTime = outReg + 1i *0;

///// save for the matlab
savemesh(Th,"diffractionSInS3d.mesh");
ffSaveVh(Th,Vh,"diffractionSInS3d_vh.txt");
ffSaveData(Ey,"diffractionSInS3d_Ey.txt");
ffSaveData(Ex1InTime,"diffractionSInS3d_Ex.txt");
ffSaveData(SigmD,"diffractionSInS3d_SigmaD.txt");
/////////////////////////
Ex1InTime = Ex + (1. - InInner()) * Ex1;
realEx1 = real(Ex1InTime);
plot(realEx1,wait=1,value=1);

real dt = 2.* pi/ w / 20;
for (int i=1; i < 100; i++)
{
t = dt * i;
cout<< "Time is" << t << endl;

Ex1InTime = Ex * exp (1i*(w*t)) + (1. - InInner()) * (Ex1 * exp (1i*(w*t)));
realEx1 = real(Ex1InTime);

plot(realEx1,wait=0,value=1);
}
\end{lstlisting}