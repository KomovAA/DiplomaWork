\chapter{Техника безопасности}

Работа не содержала экспериментальных исследований и предусматривала проведение только численного моделирования с использованием персонального компьютера.\\

\textbf{Техника безопасности при работе с компьютером}\\
1. Не трогать руками провода, электрические вилки и розетки работающего компьютера.\\
2. Запрещается  работать на компьютере мокрыми руками  или в сырой одежде.\\
3. Нельзя работать на компьютере, имеющем нарушение целостности корпуса или изоляции с неисправной индефикацией включения питания.\\
4. При появлении запаха гари или необычных звуков немедленно выключить компьютер.\\
5. При появлении  в процессе работы каких  либо неотложных дел нельзя оставлять компьютер без присмотра.  Необходимо выключить компьютер, если срок отсутствия превышает 20 мин.\\
6. Нельзя что-либо класть на компьютер, т.к. уменьшается теплоотдача металлических элементов.\\

\textbf{Требования безопасности перед работой на компьютере}\\
1. Осмотреть и привести в порядок рабочее место.\\
2. Отрегулировать освещение на рабочем месте, убедится в отсутствие потока встречного света.\\
3. Убедиться в правильности подключения электрооборудования в сети.\\
4. Протереть салфеткой поверхность экрана и защитного фильтра при наличии.\\
5. Проверить правильность установки стола и клавиатуры.\\

\textbf{Требования безопасности во время работы}\\
1. Продолжительность  работы перед экраном не должна превышать 1 часа.\\
2. В течении всего рабочего времени стол содержать в порядке.\\
3. Открыть все вентиляционные  устройства.\\
4. Выполнять санитарные нормы: соблюдать режим работы и отдыха.\\
5. Соблюдать правила эксплуатации  и вычислительной техники в соответствии с инструкциями.\\
6. Соблюдать расстояние до экрана в пределах 70–80 см.\\
7. Соблюдать установленный временем режим работы. Выполнять упражнения для рук, глаз и т.д.\\
8. Во время работы запрещается одновременно касаться экрана и клавиатуры.\\
9. Запрещается касаться задней панели системного блока при включенном питании.\\
10. Избегать попадания воды на системный блок, рабочую поверхность  и другие устройства.\\
11. Запрещается производить самостоятельное вскрытие и ремонт оборудования.\\
12. После работы на компьютере не рекомендуется  смотреть телевизор  в течение 2–3 часов.\\

\textbf{Требования безопасности в аварийных ситуациях}\\
1. Во всех случаях обрывов проводов питания, неисправности  заземления необходимо выключить компьютер.\\
2. В  случае появления рези в глазах, резком ухудшении видимости, появлении боли в пальцах немедленно покинуть рабочее место, сообщить руководителю работ и обратится к врачу.\\
3. При  возгорании оборудования  отключить питание и принять меры к тушению.